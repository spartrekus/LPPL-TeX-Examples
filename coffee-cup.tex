% Coffee cup
% Author: Mark Wibrow
\documentclass{article}
\usepackage{tikz}
%%%<
\usepackage{verbatim}
\usepackage[active,tightpage,delayed]{preview}
\PreviewEnvironment{tikzpicture}
\setlength\PreviewBorder{10pt}%
%%%>
\begin{comment}
:Title: Coffee cup
:Tags: Scopes;Postactions;Fadings;Foreach;Decorative drawings;Fun
:Author: Mark Wibrow
:Slug: coffee-cup

It may be easy to download and use a free clipart picture of a coffee cup.
However, by drawing it using TikZ we can learn something more about
TikZ' capabilities.

Here we can see the use of scopes, shifting, fadings and opacity,
and applying postactions.

The code was written by Mark Wibrow and posted on TeX.SE.
\end{comment}
\usetikzlibrary{fadings}
\tikzfading[name=fade out,
  inner color=transparent!0, outer color=transparent!100]
\begin{document}
\begin{tikzpicture}
  \foreach \c [count=\i from 0] in {white,red!75!black,blue!25,orange}{

    \tikzset{xshift={mod(\i,2)*3cm}, yshift=-floor(\i/2)*3cm}
    \colorlet{cup}{\c}

    % Saucer
    \begin{scope}[shift={(0,-1-1/16)}]
      \fill [black!87.5, path fading=fade out] 
        (0,-2/8) ellipse [x radius=6/4, y radius=3/4];
      \fill [cup, postaction={left color=black, right color=white, opacity=1/3}] 
        (0,0) ++(180:5/4) arc (180:360:5/4 and 5/8+1/16);
      \fill [cup, postaction={left color=black!50, right color=white, opacity=1/3}] 
        (0,0) ellipse [x radius=5/4, y radius=5/8];
      \fill [cup, postaction={left color=white, right color=black, opacity=1/3}]
        (0,1/16) ellipse [x radius=5/4/2, y radius=5/8/2];
      \fill [cup, postaction={left color=black, right color=white, opacity=1/3}] 
        (0,0) ellipse [x radius=5/4/2-1/16, y radius=5/8/2-1/16];
    \end{scope} 

    % Handle
    \begin{scope}[shift=(10:7/8), rotate=-30, yslant=1/2, xslant=-1/8]
      \fill [cup, postaction={top color=black, bottom color=white, opacity=1/3}] 
        (0,0) arc (130:-100:3/8 and 1/2) -- ++(0,1/4) arc (-100:130:1/8 and 1/4)
        -- cycle;
      \fill [cup, postaction={top color=white, bottom color=black, opacity=1/3}] 
        (0,0) arc (130:-100:3/8 and 1/2) -- ++(0,1/32) arc (-100:130:1/4 and 1/3)
        -- cycle;
    \end{scope}

    % Cup
    \fill [cup!25!black, path fading=fade out] 
      (0,-1-1/16) ellipse [x radius=3/4, y radius=1/3];
    \fill [cup, postaction={left color=black, right color=white, opacity=1/3/2},
      postaction={bottom color=black, top color=white, opacity=1/3/2}] 
      (-1,0) arc (180:360:1 and 5/4);
    \fill [cup, postaction={left color=white, right color=black, opacity=1/3}] 
      (0,0) ellipse [x radius=1, y radius=1/2];
    \fill [cup, postaction={left color=black, right color=white, opacity=1/3/2},
      postaction={bottom color=black, top color=white, opacity=1/3/2}] 
      (0,0) ellipse [x radius=1-1/16, y radius=1/2-1/16];

    % Coffee
    \begin{scope}
      \clip ellipse [x radius=1-1/16, y radius=1/2-1/16];
      \fill [brown!25!black] 
        (0,-1/4) ellipse [x radius=3/4, y radius=3/8];
      \fill [brown!50!black, path fading=fade out] 
        (0,-1/4) ellipse [x radius=3/4, y radius=3/8];
    \end{scope}
  }
\end{tikzpicture}
\end{document}